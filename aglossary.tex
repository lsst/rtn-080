% DO NOT EDIT - generated by /Users/womullan/LSSTgit/lsst-texmf/bin/generateAcronyms.py from https://lsst-texmf.lsst.io/.
\newacronym{AD} {AD} {Associate \gls{Director}}
\newacronym{AP} {AP} {\gls{Alert Production}}
\newacronym{APDB} {APDB} {\gls{Alert Production DataBase}}
\newacronym{API} {API} {Application Programming Interface}
\newacronym{AURA} {AURA} {\gls{Association of Universities for Research in Astronomy}}
\newglossaryentry{Alert} {name={Alert}, description={A packet of information for each source detected with signal-to-noise ratio > 5 in a difference image by Alert Production, containing measurement and characterization parameters based on the past 12 months of LSST observations plus small cutouts of the single-visit, template, and difference images, distributed via the internet}}
\newglossaryentry{Alert Production} {name={Alert Production}, description={Executing on the Prompt Processing system, the Alert Production payload processes and calibrates incoming images, performs Difference Image Analysis to identify DIASources and DIAObjects, and then packages the resulting alerts for distribution.}}
\newglossaryentry{Alert Production DataBase} {name={Alert Production DataBase}, description={A dedicated, internal database system used to support LSST Alert Production.  Does not support end-user access.}}
\newglossaryentry{Alternate Standard Visit} {name={Alternate Standard Visit}, description={A single observation of an LSST field comprised of one 30 second exposure}}
\newglossaryentry{Archive} {name={Archive}, description={The repository for documents required by the NSF to be kept. These include documents related to design and development, construction, integration, test, and operations of the LSST observatory system. The archive is maintained using the enterprise content management system DocuShare, which is accessible through a link on the project website www.project.lsst.org}}
\newglossaryentry{Archive Center} {name={Archive Center}, description={Part of the LSST Data Management System, the LSST archive center is a data center at NCSA that hosts the LSST Archive, which includes released science data and metadata, observatory and engineering data, and supporting software such as the LSST Software Stack}}
\newglossaryentry{Association Pipeline} {name={Association Pipeline}, description={An application that matches detected Sources or DIASources or generated Objects to an existing catalog of Objects, producing a (possibly many-to-many) set of associations and a list of unassociated inputs. Association Pipelines are used in Alert Production after DIASource generation and in the final stages of Data Release processing to ensure continuity of Object identifiers}}
\newglossaryentry{Association of Universities for Research in Astronomy} {name={Association of Universities for Research in Astronomy}, description={ consortium of US institutions and international affiliates that operates world-class astronomical observatories, AURA is the legal entity responsible for managing what it calls independent operating Centers, including LSST, under respective cooperative agreements with the National Science Foundation. AURA assumes fiducial responsibility for the funds provided through those cooperative agreements. AURA also is the legal owner of the AURA Observatory properties in Chile}}
\newglossaryentry{Authentication} {name={Authentication}, description={The action of demonstrating who you are and an person, mission, or other entity. Usually by use of a password or security token}}
\newacronym{B} {B} {Byte (8 bit)}
\newacronym{BCE} {BCE} {Before Common Era}
\newglossaryentry{Base Facility} {name={Base Facility}, description={The data center located at the Base Site in La Serena, Chile. The Base Facility is composed of the Base portion of the Prompt Enclave directly supporting Observatory operations, the Commissioning Cluster, an Archive Enclave holding data products, and the Chilean Data Access Center}}
\newglossaryentry{Batch Production} {name={Batch Production}, description={Computational processing that is executed as inputs become available, in a distributed way across multiple enclaves when needed, while tracking status and outputs. Examples of Batch Production include offline processing for Prompt Data Products, calibration products, template images, and Special Programs data products. Prioritization protocols for the various types of batch production are given in LDM-148}}
\newglossaryentry{Butler} {name={Butler}, description={A middleware component for persisting and retrieving image datasets (raw or processed), calibration reference data, and catalogs}}
\newacronym{CA} {CA} {Control (or Cost) Account}
\newacronym{CCB} {CCB} {\gls{Change Control Board}}
\newacronym{CCD} {CCD} {\gls{Charge-Coupled Device}}
\newacronym{CERN} {CERN} {European Organization for Nuclear Research}
\newacronym{CI} {CI} {Continuous Integration}
\newacronym{CPU} {CPU} {Central Processing Unit}
\newglossaryentry{Calibration Image} {name={Calibration Image}, description={Any of a set of images used in the Instrument Signature Removal pipeline to remove distortions caused by the telescope, detector, or other sources, from the raw images. Includes darks, flats, tunable-laser dome flats, etc}}
\newglossaryentry{Camera} {name={Camera}, description={The LSST subsystem responsible for the 3.2-gigapixel LSST camera, which will take more than 800 panoramic images of the sky every night. SLAC leads a consortium of Department of Energy laboratories to design and build the camera sensors, optics, electronics, cryostat, filters and filter exchange mechanism, and camera control system}}
\newglossaryentry{Center} {name={Center}, description={An entity managed by AURA that is responsible for execution of a federally funded project}}
\newglossaryentry{Change Control Board} {name={Change Control Board}, description={Advisory board to the Project Manager; composed of technical and management representatives who recommend approval or disapproval of proposed changes to, deviations from, and waivers to a configuration item's current approved configuration documentation}}
\newglossaryentry{Charge-Coupled Device} {name={Charge-Coupled Device}, description={a particular kind of solid-state sensor for detecting optical-band photons. It is composed of a 2-D array of pixels, and one or more read-out amplifiers}}
\newglossaryentry{Citizen Science} {name={Citizen Science}, description={the collection and analysis of data relating to the natural world by members of the general public, typically as part of a collaborative project with professional scientists.}}
\newacronym{ComCam} {ComCam} {The commissioning \gls{camera} is a single-raft, 9-CCD \gls{camera} that will be installed in LSST during commissioning, before the final \gls{camera} is ready.}
\newglossaryentry{Commissioning} {name={Commissioning}, description={A two-year phase at the end of the Construction project during which a technical team a) integrates the various technical components of the three subsystems; b) shows their compliance with ICDs and system-level requirements as detailed in the LSST Observatory System Specifications document (OSS, LSE-30); and c) performs science verification to show compliance with the survey performance specifications as detailed in the LSST Science Requirements Document (SRD, LPM-17)}}
\newglossaryentry{Construction} {name={Construction}, description={The period during which LSST observatory facilities, components, hardware, and software are built, tested, integrated, and commissioned. Construction follows design and development and precedes operations. The LSST construction phase is funded through the NSF MREFC account}}
\newacronym{DAC} {DAC} {\gls{Data Access Center}}
\newacronym{DB} {DB} {DataBase}
\newacronym{DBB} {DBB} {\gls{Data Backbone}}
\newacronym{DC2} {DC2} {Data Challenge 2 (\gls{DESC})}
\newacronym{DCR} {DCR} {\gls{Differential Chromatic Refraction}}
\newacronym{DECam} {DECam} {Dark Energy \gls{Camera}}
\newacronym{DESC} {DESC} {Dark Energy \gls{Science Collaboration}}
\newacronym{DF} {DF} {Data Facility}
\newacronym{DIA} {DIA} {\gls{Difference Image Analysis}}
\newglossaryentry{DIAObject} {name={DIAObject}, description={A DIAObject is the association of DIASources, by coordinate, that have been detected with signal-to-noise ratio greater than 5 in at least one difference image. It is distinguished from a regular Object in that its brightness varies in time, and from a SSObject in that it is stationary (non-moving)}}
\newglossaryentry{DIASource} {name={DIASource}, description={A DIASource is a detection with signal-to-noise ratio greater than 5 in a difference image}}
\newacronym{DM} {DM} {\gls{Data Management}}
\newacronym{DMS} {DMS} {\gls{Data Management Subsystem}}
\newacronym{DMS-REQ} {DMS-REQ} {Data Management System Requirements prefix}
\newacronym{DMSR} {DMSR} {DM System Requirements; \gls{LSE}-61}
\newacronym{DMTN} {DMTN} {DM Technical Note}
\newacronym{DNS} {DNS} {Domain Name Service}
\newacronym{DOE} {DOE} {\gls{Department of Energy}}
\newacronym{DP} {DP} {Data Production}
\newacronym{DP0} {DP0} {Data Preview 0}
\newacronym{DP1} {DP1} {Data Preview 1}
\newacronym{DP2} {DP2} {Data Preview 2}
\newacronym{DPDD} {DPDD} {Data Product Definition \gls{Document}}
\newacronym{DR} {DR} {\gls{Data Release}}
\newacronym{DR1} {DR1} {Data \gls{Release} 1}
\newacronym{DRP} {DRP} {\gls{Data Release Production}}
\newglossaryentry{Data Access Center} {name={Data Access Center}, description={Part of the LSST Data Management System, the US and Chilean DACs will provide authorized access to the released LSST data products, software such as the Science Platform, and computational resources for data analysis. The US DAC also includes a service for distributing bulk data on daily and annual (Data Release) timescales to partner institutions, collaborations, and LSST Education and Public Outreach (EPO). }}
\newglossaryentry{Data Backbone} {name={Data Backbone}, description={The software that provides for data registration, retrieval, storage, transport, replication, and provenance capabilities that are compatible with the Data Butler. It allows data products to move between Facilities, Enclaves, and DACs by managing caches of files at each endpoint, including persistence to long-term archival storage (e.g. tape)}}
\newglossaryentry{Data Management} {name={Data Management}, description={The LSST Subsystem responsible for the Data Management System (DMS), which will capture, store, catalog, and serve the LSST dataset to the scientific community and public. The DM team is responsible for the DMS architecture, applications, middleware, infrastructure, algorithms, and Observatory Network Design. DM is a distributed team working at LSST and partner institutions, with the DM Subsystem Manager located at LSST headquarters in Tucson}}
\newglossaryentry{Data Management Subsystem} {name={Data Management Subsystem}, description={The Data Management Subsystem is one of the four subsystems which constitute the LSST Construction Project. The Data Management Subsystem is responsible for developing and delivering the LSST Data Management System to the LSST Operations Project}}
\newglossaryentry{Data Management System} {name={Data Management System}, description={The computing infrastructure, middleware, and applications that process, store, and enable information extraction from the LSST dataset; the DMS will process peta-scale data volume, convert raw images into a faithful representation of the universe, and archive the results in a useful form. The infrastructure layer consists of the computing, storage, networking hardware, and system software. The middleware layer handles distributed processing, data access, user interface, and system operations services. The applications layer includes the data pipelines and the science data archives' products and services}}
\newglossaryentry{Data Product} {name={Data Product}, description={The LSST survey will produce three categories of Data Products. Prompt, Data Release, User Generated. Previously referred to as Levels 1, 2, and 3}}
\newglossaryentry{Data Release} {name={Data Release}, description={The approximately annual reprocessing of all LSST data, and the installation of the resulting data products in the LSST Data Access Centers, which marks the start of the two-year proprietary period}}
\newglossaryentry{Data Release Data Product} {name={Data Release Data Product}, description={These products will be made available annually as the result of coherent processing of the entire science data set to date. These will include calibrated images; measurements of positions, fluxes, and shapes; variability information such as orbital parameters for moving objects; and an appropriate compact description of light curves. The Data Release Data Products will include a uniform reprocessing of the difference-imaging-based Prompt Data Products}}
\newglossaryentry{Data Release Processing} {name={Data Release Processing}, description={Deprecated term; see Data Release Production}}
\newglossaryentry{Data Release Production} {name={Data Release Production}, description={An episode of (re)processing all of the accumulated LSST images, during which all output DR data products are generated. These episodes are planned to occur annually during the LSST survey, and the processing will be executed at the Archive Center. This includes Difference Imaging Analysis, generating deep Coadd Images, Source detection and association, creating Object and Solar System Object catalogs, and related metadata}}
\newglossaryentry{Department of Energy} {name={Department of Energy}, description={cabinet department of the United States federal government; the DOE has assumed technical and financial responsibility for providing the LSST camera. The DOE's responsibilities are executed by a collaboration led by SLAC National Accelerator Laboratory}}
\newglossaryentry{Difference Image} {name={Difference Image}, description={Refers to the result formed from the pixel-by-pixel difference of two images of the sky, after warping to the same pixel grid, scaling to the same photometric response, matching to the same PSF shape, and applying a correction for Differential Chromatic Refraction. The pixels in a difference thus formed should be zero (apart from noise) except for sources that are new, or have changed in brightness or position. In the LSST context, the difference is generally taken between a visit image and template. }}
\newglossaryentry{Difference Image Analysis} {name={Difference Image Analysis}, description={The detection and characterization of sources in the Difference Image that are above a configurable threshold, done as part of Alert Generation Pipeline}}
\newglossaryentry{Differential Chromatic Refraction} {name={Differential Chromatic Refraction}, description={The refraction of incident light by Earth's atmosphere causes the apparent position of objects to be shifted, and the size of this shift depends on both the wavelength of the source and its airmass at the time of observation. DCR corrections are done as a part of DIA}}
\newglossaryentry{Director} {name={Director}, description={The person responsible for the overall conduct of the project; the LSST director is charged with ensuring that both the scientific goals and management constraints on the project are met. S/he is the principal public spokesperson for the project in all matters and represents the project to the scientific community, AURA, the member institutions of LSSTC, and the funding agencies}}
\newglossaryentry{Docker} {name={Docker}, description={A system for packaging and distributing software using self-contained containers which may be run on any Linux system; \url{https://www.docker.com/}}}
\newglossaryentry{DocuShare} {name={DocuShare}, description={The trade name for the enterprise management software used by LSST to archive and manage documents}}
\newglossaryentry{Document} {name={Document}, description={Any object (in any application supported by DocuShare or design archives such as PDMWorks or GIT) that supports project management or records milestones and deliverables of the LSST Project}}
\newacronym{EFD} {EFD} {Engineering and Facility Database}
\newacronym{EPO} {EPO} {\gls{Education and Public Outreach}}
\newacronym{ESNet} {ESNet} {Energy Sciences Network}
\newglossaryentry{Education and Public Outreach} {name={Education and Public Outreach}, description={The LSST subsystem responsible for the cyberinfrastructure, user interfaces, and outreach programs necessary to connect educators, planetaria, citizen scientists, amateur astronomers, and the general public to the transformative LSST dataset}}
\newglossaryentry{Enclave} {name={Enclave}, description={Individually defined portions of the computational resources at the Summit, Base, NCSA, and Satellite Facilities, such as the Prompt Enclave, the Archive Enclave, etc. }}
\newacronym{FITS} {FITS} {\gls{Flexible Image Transport System}}
\newacronym{FLOPS} {FLOPS} {FLoating point Operation per Second}
\newacronym{FOA} {FOA} {Funding \gls{Opportunity} Announcement}
\newacronym{FY23} {FY23} {Financial Year 23}
\newglossaryentry{Flexible Image Transport System} {name={Flexible Image Transport System}, description={an international standard in astronomy for storing images, tables, and metadata in disk files. See the IAU FITS Standard for details}}
\newglossaryentry{ForcedSource} {name={ForcedSource}, description={DRP table resulting from forced photometry}}
\newacronym{GB} {GB} {Gigabyte}
\newacronym{GID} {GID} {Group Identifier}
\newacronym{GNU} {GNU} {GNU's Not Unix! An operating system and an extensive collection of free computer \gls{software}}
\newacronym{GPL} {GPL} {GNU Public License}
\newacronym{Gb} {Gb} {Gigabit}
\newacronym{HSC} {HSC} {Hyper Suprime-Cam}
\newglossaryentry{Handle} {name={Handle}, description={The unique identifier assigned to a document uploaded to DocuShare}}
\newacronym{IAU} {IAU} {International Astronomical Union}
\newacronym{ICD} {ICD} {\gls{Interface Control Document}}
\newacronym{IN2P3} {IN2P3} {Institut National de Physique Nucléaire et de Physique des Particules}
\newacronym{IP} {IP} {Internet Protocol}
\newacronym{IT} {IT} {Information Technology}
\newacronym{ITC} {ITC} {Information Technology \gls{Center}}
\newacronym{IVOA} {IVOA} {International Virtual-Observatory Alliance}
\newglossaryentry{Instrument Signature Removal} {name={Instrument Signature Removal}, description={Instrument Signature Removal is a pipeline that applies calibration reference data in the course of raw data processing, to remove artifacts of the instrument or detector electronics, such as removal of overscan pixels, bias correction, and the application of a flat-field to correct for pixel-to-pixel variations in sensitivity}}
\newglossaryentry{Interface Control Document} {name={Interface Control Document}, description={A Document that describes, defines, and controls the interface(s) of a system, thereby bounding its requirements. The description includes the inputs and outputs of a single system or element. An ICD may also describe the interface between two systems or subsystems. The purpose of the ICD is to communicate all possible inputs to and all potential outputs from a system for some potential or actual user of the system in operations. The internal interfaces of a system or subsystem are typically not documented in an ICD, but rather in a system design document}}
\newglossaryentry{J2000} {name={J2000}, description={Julian Date referring to the instant of 12 noon (midday) on January 1, 2000. IAU standard equinox.}}
\newacronym{JD} {JD} {\gls{Julian Date}}
\newglossaryentry{JIRA} {name={JIRA}, description={issue tracking product (not an acronym but a truncation of Gojira the Japanese name for Godzilla)}}
\newglossaryentry{Julian Date} {name={Julian Date}, description={The Julian Date (JD) of any instant is the Julian day number for the preceding noon (UTC), plus the fraction of the day elapsed since that instant. The Julian day number is a running sequence of integral days, starting at noon, since the beginning of the Julian Period; JD 0.0 corresponds to noon on 1 January 4713 BCE. Various Julian Date converters are available on the Web. For example, 18h 00m 00.0s UT on 2014-July-01 (near the start of LSST construction) corresponds to JD 2456840.25}}
\newacronym{K8S} {K8S} {Kubernetes provisioning system}
\newacronym{KB} {KB} {KiloByte}
\newglossaryentry{Kubernetes} {name={Kubernetes}, description={A system for automating application deployment and management using software containers (e.g. Docker); \url{https://kubernetes.io}}}
\newacronym{LATISS} {LATISS} {LSST Atmospheric Transmission Imager and Slitless Spectrograph}
\newacronym{LCR} {LCR} {\gls{LSST Change Request}}
\newacronym{LDF} {LDF} {LSST Data Facility}
\newacronym{LDM} {LDM} {LSST Data Management (Document \gls{Handle})}
\newacronym{LDO} {LDO} {LSST Document \gls{Operations} (Document Handle)}
\newacronym{LHN} {LHN} {Long Haul Network}
\newacronym{LOY1} {LOY1} {LSST \gls{Operations} Year 1}
\newacronym{LPM} {LPM} {LSST Project Management (Document \gls{Handle})}
\newacronym{LSE} {LSE} {LSST \gls{Systems Engineering} (Document Handle)}
\newacronym{LSP} {LSP} {LSST \gls{Science Platform} (now Rubin \gls{Science Platform})}
\newacronym{LSST} {LSST} {Legacy Survey of Space and Time (formerly Large Synoptic Survey Telescope)}
\newglossaryentry{LSST Change Request} {name={LSST Change Request}, description={document that proposes a change to a configuration item; after evaluation by the CCB and decision by the Project Manager, the change request is updated with the outcome, action items, and necessary notification}}
\newglossaryentry{LSST Corporation} {name={LSST Corporation}, description={An Arizona 501(c)3 not-for-profit corporation formed in 2003 for the purpose of designing, constructing, and operating the LSST System. During design and development, the Corporation stewarded private funding used for such essential contributions as early site preparation, mirror construction, and early data management system development. During construction, LSSTC will secure private operations funding from international affiliates and play a key role in preparing the scientific community to use the LSST dataset}}
\newglossaryentry{LSST Project Office} {name={LSST Project Office}, description={Official name of the stand-alone AURA operating center responsible for execution of the LSST construction project under the NSF MREFC account}}
\newglossaryentry{LSST Science Pipelines} {name={LSST Science Pipelines}, description={software used to perform the LSST data reduction pipelines.lsst.io}}
\newacronym{LSSTC} {LSSTC} {\gls{LSST Corporation}}
\newacronym{LSSTPO} {LSSTPO} {\gls{LSST Project Office}}
\newglossaryentry{Level 1 Data Product} {name={Level 1 Data Product}, description={Deprecated term; see Prompt Data Product}}
\newglossaryentry{Level 3 Data Product} {name={Level 3 Data Product}, description={Deprecated term; see User Generated Data Product}}
\newacronym{MB} {MB} {MegaByte}
\newacronym{MPC} {MPC} {Minor Planet \gls{Center}}
\newacronym{MREFC} {MREFC} {\gls{Major Research Equipment and Facility Construction}}
\newglossaryentry{Major Research Equipment and Facility Construction} {name={Major Research Equipment and Facility Construction}, description={the NSF account through which large facilities construction projects such as LSST are funded}}
\newglossaryentry{Mapper} {name={Mapper}, description={A piece of software that abstracts persisting and unpersisting data; specifically, it knows how to navigate a data repository to locate data that match selection criteria that are relevant for data obtained with a particular camera. Used by the Butler}}
\newacronym{NCSA} {NCSA} {National \gls{Center} for Supercomputing Applications}
\newacronym{NET} {NET} {Network Engineering Team}
\newacronym{NFS} {NFS} {Network File System}
\newacronym{NOIRLab} {NOIRLab} {NSF's National Optical-Infrared Astronomy Research Laboratory; \url{https://noirlab.edu}}
\newacronym{NSF} {NSF} {\gls{National Science Foundation}}
\newglossaryentry{National Science Foundation} {name={National Science Foundation}, description={primary federal agency supporting research in all fields of fundamental science and engineering; NSF selects and funds projects through competitive, merit-based review}}
\newglossaryentry{Non-Standard Visit} {name={Non-Standard Visit}, description={Any single observation of a LSST field that is not comprised of either two 15 second 'Snap' exposures (a standard visit) or one 30 second exposure (an alternative standard visit). For example, exposure times for Special Programs might be significantly shorter or longer than a standard visit (or of random length)}}
\newacronym{OCS} {OCS} {Observatory Control System}
\newacronym{OPS} {OPS} {\gls{Operations}}
\newacronym{OSS} {OSS} {Observatory System Specifications; \gls{LSE}-30}
\newglossaryentry{Object} {name={Object}, description={In LSST nomenclature this refers to an astronomical object, such as a star, galaxy, or other physical entity. E.g., comets, asteroids are also Objects but typically called a Moving Object or a Solar System Object (SSObject). One of the DRP data products is a table of Objects detected by LSST which can be static, or change brightness or position with time}}
\newglossaryentry{Operations} {name={Operations}, description={The 10-year period following construction and commissioning during which the LSST Observatory conducts its survey}}
\newglossaryentry{Opportunity} {name={Opportunity}, description={The degree of exposure to an event that might happen to the benefit of a program, project, or other activity. It is described by a combination of the probability that the opportunity event will occur and the consequence of the extent of gain from the occurrence, or impact. There are two levels of opportunities. At the macro level, a project itself is the manifestation of the pursuit of an opportunity. At the element level, tactical opportunities exist, whereby certain events, if realized, provide a cost or schedule savings to the project or increase technical performance}}
\newacronym{PB} {PB} {PetaByte}
\newacronym{PCI} {PCI} {Peripheral Component Interconnect}
\newacronym{PDR1} {PDR1} {Public Data \gls{Release} 1 (HSC)}
\newacronym{PDR2} {PDR2} {Public Data \gls{Release} 2 (HSC)}
\newacronym{POSIX} {POSIX} {Portable Operating System Interface}
\newacronym{PPDB} {PPDB} {\gls{Prompt Products DataBase}}
\newacronym{PSF} {PSF} {Point Spread Function}
\newacronym{PVI} {PVI} {\gls{Processed Visit Image}}
\newglossaryentry{Processed Visit Image} {name={Processed Visit Image}, description={A calibrated, background-subtracted image from a single visit, packaged with quality mask and variance arrays, as well as a PSF characterization, detailed calibration data, and other metadata about the image.}}
\newglossaryentry{Project Manager} {name={Project Manager}, description={The person responsible for exercising leadership and oversight over the entire Rubin project; he or she controls schedule, budget, and all contingency funds}}
\newglossaryentry{Prompt Data Product} {name={Prompt Data Product}, description={Prompt Data Products are generated continuously based on the image stream from the telescope by the Prompt Processing system. They include low-latency alerts on transient and variable sources, as well as a variety of image data products and source catalogs. Compare Data Release Data Product.}}
\newglossaryentry{Prompt Processing} {name={Prompt Processing}, description={The data processing which occurs at the Archive Center based on the stream of images coming from the telescope. This includes both Alert Production, which scans the image stream to identify and send alerts on transient and variable sources, and Solar System Processing, which identifies and characterizes objects in our solar system. It also includes specialized rapid calibration and Commissioning processing. Prompt Processing generates the Prompt Data Products.}}
\newglossaryentry{Prompt Products DataBase} {name={Prompt Products DataBase}, description={Data products within LSST data releases relating to LSST Alert Production}}
\newacronym{QA} {QA} {\gls{Quality Assurance}}
\newacronym{QC} {QC} {\gls{Quality Control}}
\newglossaryentry{Qserv} {name={Qserv}, description={LSST's distributed parallel database. This database system is used for collecting, storing, and serving LSST Data Release Catalogs and Project metadata, and is part of the Software Stack}}
\newglossaryentry{Quality Assurance} {name={Quality Assurance}, description={All activities, deliverables, services, documents, procedures or artifacts which are designed to ensure the quality of DM deliverables. This may include QC systems, in so far as they are covered in the charge described in LDM-622. Note that contrasts with the LDM-522 definition of “QA” as “Quality Analysis”, a manual process which occurs only during commissioning and operations. See also: Quality Control}}
\newglossaryentry{Quality Control} {name={Quality Control}, description={Services and processes which are aimed at measuring and monitoring a system to verify and characterize its performance (as in LDM-522). Quality Control systems run autonomously, only notifying people when an anomaly has been detected. See also Quality Assurance}}
\newacronym{RAM} {RAM} {Random Access Memory}
\newacronym{RFC} {RFC} {Request For Comment}
\newacronym{RSP} {RSP} {Rubin \gls{Science Platform}}
\newacronym{RTN} {RTN} {Rubin Technical Note}
\newglossaryentry{Release} {name={Release}, description={Publication of a new version of a document, software, or data product. Depending on context, releases may require approval from Project- or DM-level change control boards, and then form part of the formal project baseline}}
\newglossaryentry{Review} {name={Review}, description={Programmatic and/or technical audits of a given component of the project, where a preferably independent committee advises further project decisions, based on the current status and their evaluation of it. The reviews assess technical performance and maturity, as well as the compliance of the design and end product with the stated requirements and interfaces}}
\newglossaryentry{Risk} {name={Risk}, description={The degree of exposure to an event that might happen to the detriment of a program, project, or other activity. It is described by a combination of the probability that the risk event will occur and the consequence of the extent of loss from the occurrence, or impact. Risk is an inherent part of all activities, whether the activity is simple and small, or large and complex}}
\newglossaryentry{Rubin Operations} {name={Rubin Operations}, description={operations phase of Vera C. Rubin Observatory}}
\newacronym{Rucio} {Rucio} {A scientific data management system developed at \gls{CERN}; \url{https://rucio.cern.ch}}
\newacronym{S3} {S3} {(Amazon) Simple Storage Service}
\newacronym{SIA} {SIA} {Simple Image Access (\gls{IVOA} standard)}
\newacronym{SLA} {SLA} {Service Level Agreement}
\newacronym{SLAC} {SLAC} {\gls{SLAC National Accelerator Laboratory}}
\newglossaryentry{SLAC National Accelerator Laboratory} {name={SLAC National Accelerator Laboratory}, description={A national laboratory funded by the US Department of Energy (DOE); SLAC leads a consortium of DOE laboratories that has assumed responsibility for providing the LSST camera. Although the Camera project manages its own schedule and budget, including contingency, the Camera team’s schedule and requirements are integrated with the larger Project.  The camera effort is accountable to the LSSTPO.}}
\newacronym{SODA} {SODA} {Server-side \gls{Operations} for Data Access (IVOA standard)}
\newacronym{SOW} {SOW} {Statement Of Work}
\newacronym{SQL} {SQL} {Structured Query Language}
\newacronym{SQuaSH} {SQuaSH} {\gls{Science Quality Analysis Harness}}
\newacronym{SRD} {SRD} {LSST Science Requirements; \gls{LPM}-17}
\newacronym{SSD} {SSD} {Solid-State Disk}
\newacronym{SSL} {SSL} {Secure Sockets Layer}
\newacronym{SSP} {SSP} {\gls{Solar System Processing}}
\newglossaryentry{Science Collaboration} {name={Science Collaboration}, description={An autonomous body of scientists interested in a particular area of science enabled by the LSST dataset, which through precursor studies, simulations, and algorithm development lays the groundwork for the large-scale science projects the LSST will enable.  In addition to preparing their members to take full advantage of LSST early in its operations phase, the science collaborations have helped to define the system's science requirements, refine and promote the science case, and quality check design and development work}}
\newglossaryentry{Science Pipelines} {name={Science Pipelines}, description={The library of software components and the algorithms and processing pipelines assembled from them that are being developed by DM to generate science-ready data products from LSST images. The Pipelines may be executed at scale as part of LSST Prompt or Data Release processing, or pieces of them may be used in a standalone mode or executed through the Rubin Science Platform. The Science Pipelines are one component of the LSST Software Stack}}
\newglossaryentry{Science Platform} {name={Science Platform}, description={A set of integrated web applications and services deployed at the LSST Data Access Centers (DACs) through which the scientific community will access, visualize, and perform next-to-the-data analysis of the LSST data products}}
\newglossaryentry{Science Quality Analysis Harness} {name={Science Quality Analysis Harness}, description={provides a minimal infrastructure for monitoring the LSST verification metrics. It can be used and extended to preserve the code and knowledge developed during LSST construction \url{https://squash.lsst.codes/}}}
\newglossaryentry{Scope} {name={Scope}, description={The work needed to be accomplished in order to deliver the product, service, or result with the specified features and functions}}
\newglossaryentry{Snap} {name={Snap}, description={One 15 second exposure within a Standard Visit in the LSST cadence}}
\newglossaryentry{Software Stack} {name={Software Stack}, description={Often referred to as the LSST Stack, or just The Stack, it is the collection of software written by the LSST Data Management Team to process, generate, and serve LSST images, transient alerts, and catalogs. The Stack includes the LSST Science Pipelines, as well as packages upon which the DM software depends. It is open source and publicly available}}
\newglossaryentry{Solar System Object} {name={Solar System Object}, description={A solar system object is an astrophysical object that is identified as part of the Solar System: planets and their satellites, asteroids, comets, etc. This class of object had historically been referred to within the LSST Project as Moving Objects}}
\newglossaryentry{Solar System Processing} {name={Solar System Processing}, description={A component of the Prompt Processing system, Solar System Processing identifies new SSObjects using unassociated DIASources.}}
\newglossaryentry{Source} {name={Source}, description={A single detection of an astrophysical object in an image, the characteristics for which are stored in the Source Catalog of the DRP database. The association of Sources that are non-moving lead to Objects; the association of moving Sources leads to Solar System Objects. (Note that in non-LSST usage "source" is often used for what LSST calls an Object.)}}
\newglossaryentry{Special Program} {name={Special Program}, description={Any LSST mini-survey or deep drilling field that is observed independently of the Wide-Fast-Deep (WFD) main survey}}
\newglossaryentry{Specification} {name={Specification}, description={One or more performance parameter(s) being established by a requirement that the delivered system or subsystem must meet}}
\newglossaryentry{Standard Visit} {name={Standard Visit}, description={A single observation of a LSST field comprised of two 15 second 'Snap' exposures that are immediately combined. An 'Alternate Standard Visit' is a single observation of a LSST field comprised of one 30 second exposure}}
\newglossaryentry{Subsystem} {name={Subsystem}, description={A set of elements comprising a system within the larger LSST system that is responsible for a key technical deliverable of the project}}
\newglossaryentry{Subsystem Manager} {name={Subsystem Manager}, description={responsible manager for an LSST subsystem; he or she exercises authority, within prescribed limits and under scrutiny of the Project Manager, over the relevant subsystem's cost, schedule, and work plans}}
\newglossaryentry{Summit} {name={Summit}, description={The site on the Cerro Pach\'{o}n, Chile mountaintop where the LSST observatory, support facilities, and infrastructure will be built}}
\newglossaryentry{Summit Facility} {name={Summit Facility}, description={The main Observatory and Auxiliary Telescope buildings at the Summit Site on Cerro Pach\'{o}n, Chile}}
\newglossaryentry{Systems Engineering} {name={Systems Engineering}, description={an interdisciplinary field of engineering that focuses on how to design and manage complex engineering systems over their life cycles. Issues such as requirements engineering, reliability, logistics, coordination of different teams, testing and evaluation, maintainability and many other disciplines necessary for successful system development, design, implementation, and ultimate decommission become more difficult when dealing with large or complex projects. Systems engineering deals with work-processes, optimization methods, and risk management tools in such projects. It overlaps technical and human-centered disciplines such as industrial engineering, control engineering, software engineering, organizational studies, and project management. Systems engineering ensures that all likely aspects of a project or system are considered, and integrated into a whole}}
\newacronym{TAP} {TAP} {Table Access Protocol (\gls{IVOA} standard)}
\newacronym{TB} {TB} {TeraByte}
\newacronym{TLS} {TLS} {Transport Layer Security}
\newacronym{TS} {TS} {Test \gls{Specification}}
\newacronym{UID} {UID} {User Identifier}
\newacronym{US} {US} {United States}
\newacronym{USDF} {USDF} {US Data Facility}
\newacronym{UT} {UT} {Universal Time}
\newacronym{UTC} {UTC} {Coordinated Universal Time}
\newglossaryentry{User Generated Data Product} {name={User Generated Data Product}, description={The products of User Generated Processing pipelines; these products will originate from the community, including project teams}}
\newglossaryentry{User Generated Processing} {name={User Generated Processing}, description={Any (re)processing of LSST data performed by a user, with either custom pipelines or reconfigured LSST software, to create User Generated Data Products. This processing will originate from the community, including project teams}}
\newacronym{VPN} {VPN} {Virtual Private Network}
\newglossaryentry{Visit} {name={Visit}, description={A sequence of one or more consecutive exposures at a given position, orientation, and filter within the LSST cadence. See Standard Visit, Alternate Standard Visit, and Non-Standard Visit}}
\newacronym{WBS} {WBS} {\gls{Work Breakdown Structure}}
\newacronym{WFD} {WFD} {Wide Fast Deep}
\newacronym{WP} {WP} {Work Package}
\newglossaryentry{Wide-Fast-Deep} {name={Wide-Fast-Deep}, description={The main survey of the LSST to cover at least 18000 square degrees of the southern sky}}
\newglossaryentry{Work Breakdown Structure} {name={Work Breakdown Structure}, description={a tool that defines and organizes the LSST project's total work scope through the enumeration and grouping of the project's discrete work elements}}
\newglossaryentry{airmass} {name={airmass}, description={The pathlength of light from an astrophysical source through the Earth's atmosphere. It is given approximately by sec z, where z is the angular distance from the zenith (the point directly overhead, where airmass = 1.0) to the source}}
\newglossaryentry{algorithm} {name={algorithm}, description={A computational implementation of a calculation or some method of processing}}
\newglossaryentry{astronomical object} {name={astronomical object}, description={A star, galaxy, asteroid, or other physical object of astronomical interest. Beware: in non-LSST usage, these are often known as sources}}
\newglossaryentry{background} {name={background}, description={In an image, the background consists of contributions from the sky (e.g., clouds or scattered moonlight), and from the telescope and camera optics, which must be distinguished from the astrophysical background. The sky and instrumental backgrounds are characterized and removed by the LSST processing software using a low-order spatial function whose coefficients are recorded in the image metadata}}
\newglossaryentry{cadence} {name={cadence}, description={The sequence of pointings, visit exposures, and exposure durations performed over the course of a survey}}
\newglossaryentry{calibration} {name={calibration}, description={The process of translating signals produced by a measuring instrument such as a telescope and camera into physical units such as flux, which are used for scientific analysis. Calibration removes most of the contributions to the signal from environmental and instrumental factors, such that only the astronomical component remains}}
\newglossaryentry{camera} {name={camera}, description={An imaging device mounted at a telescope focal plane, composed of optics, a shutter, a set of filters, and one or more sensors arranged in a focal plane array}}
\newglossaryentry{cloud} {name={cloud}, description={A visible mass of condensed water vapor floating in the atmosphere, typically high above the ground or in interstellar space acting as the birthplace for stars.  Also a way of computing (on other peoples computers leveraging their services and availability).}}
\newglossaryentry{configuration} {name={configuration}, description={A task-specific set of configuration parameters, also called a 'config'. The config is read-only; once a task is constructed, the same configuration will be used to process all data. This makes the data processing more predictable: it does not depend on the order in which items of data are processed. This is distinct from arguments or options, which are allowed to vary from one task invocation to the next}}
\newglossaryentry{cycle} {name={cycle}, description={The time period over which detailed, short-term plans are defined and executed. Normally, cycles run for six months, and culminate in a new release of the LSST Software Stack, however this need not always be the case}}
\newglossaryentry{data collection} {name={data collection}, description={A data collection in the second-generation (Gen2) Butler (referred to as a data repository in earlier generations) consists of hierarchically organized data files, an inventory or registry of the contents (i.e., metadata from the data files) stored in an sqlite3 file, and a Mapper file that specifies to the LSST Stack software the camera model to apply when accessing the data in the data repository}}
\newglossaryentry{data repository} {name={data repository}, description={A data repository consists of hierarchically organized data files, an inventory or registry of the contents (i.e., metadata from the data files) stored in an sqlite3 file, and a Mapper file that specifies to the LSST Stack software the camera model to apply when accessing the data in the repository. With the second-generation (Gen2) Butler, the term repository will be replaced by data collection}}
\newglossaryentry{element} {name={element}, description={A node in the hierarchical project WBS}}
\newglossaryentry{epoch} {name={epoch}, description={Sky coordinate reference frame, e.g., J2000. Alternatively refers to a single observation (usually photometric, can be multi-band) of a variable source}}
\newglossaryentry{flux} {name={flux}, description={Shorthand for radiative flux, it is a measure of the transport of radiant energy per unit area per unit time. In astronomy this is usually expressed in cgs units: erg/cm2/s}}
\newglossaryentry{forced photometry} {name={forced photometry}, description={A measurement of the photometric properties of a source, or expected source, with one or more parameters held fixed. Most often this means fixing the location of the center of the brightness profile (which may be known or predicted in advance), and measuring other properties such as total brightness, shape, and orientation. Forced photometry will be done for all Objects in the Data Release Production}}
\newglossaryentry{metadata} {name={metadata}, description={General term for data about data, e.g., attributes of astronomical objects (e.g. images, sources, astroObjects, etc.) that are characteristics of the objects themselves, and facilitate the organization, preservation, and query of data sets. (E.g., a FITS header contains metadata)}}
\newglossaryentry{middleware} {name={middleware}, description={Software that acts as a bridge between other systems or software usually a database or network. Specifically in the Data Management System this refers to Butler for data access and Workflow management for distributed processing.}}
\newglossaryentry{monitoring} {name={monitoring}, description={In DM QA, this refers to the process of collecting, storing, aggregating and visualizing metrics}}
\newglossaryentry{passband} {name={passband}, description={The window of wavelength or the energy range admitted by an optical system; specifically the transmission as a function of wavelength or energy. Typically the passband is limited by a filter. The width of the passband may be characterized in a variety of ways, including the width of the half-power points of the transmission curve, or by the equivalent width of a filter with 100\% transmission within the passband, and zero elsewhere}}
\newglossaryentry{patch} {name={patch}, description={An quadrilateral sub-region of a sky tract, with a size in pixels chosen to fit easily into memory on desktop computers}}
\newglossaryentry{pipeline} {name={pipeline}, description={A configured sequence of software tasks (Stages) to process data and generate data products. Example: Association Pipeline}}
\newglossaryentry{provenance} {name={provenance}, description={Information about how LSST images, Sources, and Objects were created (e.g., versions of pipelines, algorithmic components, or templates) and how to recreate them}}
\newglossaryentry{schema} {name={schema}, description={The definition of the metadata and linkages between datasets and metadata entities in a collection of data or archive.}}
\newglossaryentry{seeing} {name={seeing}, description={An astronomical term for characterizing the stability of the atmosphere, as measured by the width of the point-spread function on images. The PSF width is also affected by a number of other factors, including the airmass, passband, and the telescope and camera optics}}
\newglossaryentry{shape} {name={shape}, description={In reference to a Source or Object, the shape is a functional characterization of its spatial intensity distribution, and the integral of the shape is the flux. Shape characterizations are a data product in the DIASource, DIAObject, Source, and Object catalogs}}
\newglossaryentry{sky map} {name={sky map}, description={A sky tessellation for LSST. The Stack includes software to define a geometric mapping from the representation of World Coordinates in input images to the LSST sky map. This tessellation is comprised of individual tracts which are, in turn, comprised of patches}}
\newglossaryentry{software} {name={software}, description={The programs and other operating information used by a computer.}}
\newglossaryentry{sqlite3} {name={sqlite3}, description={A software package external to DM, sqlite3 provides a SQL interface compliant with the DB-API 2.0 specification for SQLite, a self-contained public-domain SQL database engine}}
\newglossaryentry{stack} {name={stack}, description={a grouping, usually in layers (hence stack), of software packages and services to achieve a common goal. Often providing a higher level set of end user oriented services and tools}}
\newglossaryentry{tract} {name={tract}, description={A portion of sky, a spherical convex polygon, within the LSST all-sky tessellation (sky map). Each tract is subdivided into sky patches}}
\newglossaryentry{transient} {name={transient}, description={A transient source is one that has been detected on a difference image, but has not been associated with either an astronomical object or a solar system body}}
