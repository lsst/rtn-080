\section{Data Management System portability and cloud computing}
\label{sec:studies}

Although the project baseline has physical compute facilities in Chile and \gls{USDF}, with split-site processing at \gls{IN2P3}, we have designed the system to be flexible with regard to the environment within which it is deployed.
It is also possible the computing systems at the \gls{USDF} and Chilean Access Centers will not satisfy the entire demand for near-the-data processing and peak load; computing models that allow externally-funded resources to be easily and efficiently used are desirable.
Public clouds, by handling the accounting and resource management for multi-tenancy, provide an interesting solution for this, provided that potentially-large data storage costs can be mitigated. Any \gls{USDF} partner shall be open to continuing investigations and partnerships of this type in collaboration with
Rubin Observatory.

In this vein, we have recently undertaken studies to investigate the possibility of performing \gls{LSST} data processing on cloud computing platforms provided by both Google and Amazon.
On the Google cloud platform, we demonstrated that several of the major components of the \gls{Data Management System} could be run effectively.
In particular,

\begin{itemize}

  \item{we deployed the \gls{Qserv} database system, demonstrating that it could achieve 80\% of the performance we achieved in-house on physical hardware;}

  \item{we demonstrated data transfer adequate for \gls{Prompt Processing}, within the limits of the current \gls{LHN} networks available for testing;}.

  \item{we deployed and tested the Prompt Products Database;}

  \item{we deployed an instance of the \gls{LSP}.}

\end{itemize}

Note that the \gls{LSP} in particular is engineered around the \gls{K8S} system (\secref{sec:k8s}), and therefore deploys extremely smoothly to the Google Cloud.
The results of this study are described in detail in \citeds{DMTN-125}.

The Google study did not investigate the single largest compute load that the \gls{LSST} will face: \gls{Data Release Processing}.
We are now addressing this on Amazon Web Services / Elastic Compute Cloud, as described in \citeds{DMTN-114}.
This work is ongoing at time of writing; initial results are positive.

We believe that these studies demonstrate the flexibility of the \gls{DM} System to a variety of deployment environments, and particularly illustrate the value --- and importance to the project of --- \gls{K8S}.
