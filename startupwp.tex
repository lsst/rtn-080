\subsection{\textbf{WP-00}: \gls{USDF} Startup}
\label{sect:wp00}

The \gls{USDF} is expected to start operations in late FY23 or early FY24.
There will need to be overlap with construction and interim operations
facilities in FY23 and
possibly sooner to enable a smooth and successful transition to the \gls{USDF}.

Developer support is one aspect of this transition -- there need to be several
machines
similar to the target machines for checking code and testing. Such transition
activities
will involve close conversations between the \gls{USDF} awardee and Rubin staff; however, a first estimate would be to provide similar
machines to those defined in
\url{https://developer.lsst.io/services/lsst-dev.html}.

In the following, requirements are called out in {\bf bold face} with integer indices for each type of requirement. The first set listed below are precursor requirements that establish the minimum capability of the \gls{USDF} before operations of the full \gls{LSST} begins.

\newreqtype{INIT}
\reqsimp{}{}{}{}{}
{
At least one year before \gls{LOY1} the USDF shall have developer support setup. This should include
services similar to the LDF services listed on \url{https://developer.lsst.io/services}.
}

\reqsimp{}{}{}{}{}
{
At least one year before LOY1 the USDF shall have storage resources available for development and prototypical Data Access services.
At this time, we estimate that 5 PB of object, 9PB of normal and 16 PB of tape storage shall be available
 (See also \tabref{tab:foa22}.)
}

\tiny \begin{longtable} { |p{0.22\textwidth}  |r  |r |} 
\caption{This table outlines the estimated needs pre LOY1  \label{tab:foa22}}\\ 
\hline 
\textbf{Year}&\textbf{2022} \\ \hline
{Instantaneous cores (DRP) Total}&{4,673} \\ \hline
{Instantaneous cores (Alerts)}&{1188} \\ \hline
{Qserv nodes (US DAC/ Staff)}&{14} \\ \hline
\textbf{Total owned nodes}&\textbf{567} \\ \hline
{Fast Storage (TB)}&{50} \\ \hline
{Normal Storage (TB)}&{9241} \\ \hline
{Latent Storage  (TB)}&{4966} \\ \hline
{High Latency (TB)}&{16733} \\ \hline
\end{longtable} \normalsize


For \tabref{tab:foa22} the storage descriptions are given in \tabref{tab:idfstorage} and
the machines are described in \tabref{tab:idfmachines}.

\tiny \begin{longtable} { |p{0.22\textwidth}  |r  |r  |r  |r |} 
\caption{Machine types used in sizing model. \label{tab:idfmachines}}\\ 
\hline 
\textbf{Type of machine }&\textbf{Cores}&\textbf{Memory(GB)}&\textbf{Eff cores/ node} \\ \hline
{Xeon }&{32}&{192}&{27} \\ \hline
{Qserv }&{12}&{128}&{12} \\ \hline
{current compute node }&{24}&{128}&{24} \\ \hline
\end{longtable} \normalsize

The available memory per core should average at least 6 GB, with 10 GB for Qserv cores.
The actual number of cores and thus core-hours may need to be adjusted based on performance of the selected node type.


\tiny \begin{longtable} { |p{0.22\textwidth}  |r |} 
\caption{Storage types used in sizing model. \label{tab:idfstorage}}\\ 
\hline 
\textbf{Storage type } \\ \hline
{fast -- NVMe (50GB/ s each) / TB  } \\ \hline
{normal - SATA GPFS file systems/ TB  } \\ \hline
{latency -- slower but on disk } \\ \hline
{high latency -- very slow -- on tape } \\ \hline
\end{longtable} \normalsize


\subsection{Networking} \label{sec:networking}
The distributed nature of the LSST facilities necessitates a complex network design, which is described in \citeds{LSE-78}.
This includes local networking at the Summit Facility, at the Base Site, and at the Data Facility, as well as long-haul networks connecting Chile and North America.

\reqsimp{}{}{}{}{}
{
At least six months before LOY1 the USDF shall have arranged 100Gbs, path redundant, network capacity to \gls{ESNet} or equivalent to connect to the  Rubin Observatory facility in Chile.
}

\reqsimp{}{}{}{}{}
{The USDF shall ensure the capacity to IN2P3 is not impeded on the US side. IN2P3 are responsible for the transatlantic transfer of data.
}

\reqsimp{}{}{}{}{}
{The USDF shall ensure low latency on the links to Chile to enable alert processing within 60s. The latency due to the network shall be  3s or better - the USDF must ensure their contribution to latency is within this envelope.
\label{req:netlat}
}

\reqsimp{}{}{}{}{}
{
        The facility shall reserve an allocation of 10\,Gb/s for alert stream transfer from the \gls{USDF} to community brokers; see \citeds{LDM-612}. \label{req:netbroker}
}

\reqsimp{}{}{}{}{}
{
The USDF shall provide high-performance (100\,Gb/s) network paths between cluster compute nodes.
}
We expect that detailed tuning of the cluster networking system may be necessary for optimal performance, and would require support for this from the Data Facility host.
\reqsimp{}{}{}{}{}
{
The USDF shall provide specific network overlays, such as Weave, if required for the Kubernetes cluster (see \reqref{req:k8s}).

\reqsimp{}{}{}{}{}{}
{
The USDF shall allow Rubin to partition the infrastructure network from the Internet so that internal services and administrative interfaces are not generally accessible.
}
\reqsimp{}{}{}{}{}{}
{
 The USDF shall enable Rubin to define Internet ingresses, including  a mechanism for engineers to access the internal network and administrative interfaces, via a bastion host, VPN, or other similar mechanism.
}


